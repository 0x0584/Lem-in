%******************************************************************************%
%                                                                              %
%                                                         :::      ::::::::    %
%    draft.tex                                          :+:      :+:    :+:    %
%                                                     +:+ +:+         +:+      %
%    By: archid- <archid-@student.1337.ma>          +#+  +:+       +#+         %
%                                                 +#+#+#+#+#+   +#+            %
%    Created: 2019/12/27 01:23:21 by archid-           #+#    #+#              %
%    Updated: 2019/12/27 03:03:57 by archid-          ###   ########.fr        %
%                                                                              %
%******************************************************************************%

\documentclass{article}
\usepackage[utf8]{inputenc}
\usepackage[a4paper, total={6in, 8in}]{geometry}

\title{Draft: Synchronization and Flow Networks}
\author{R. Anas.}
\date{December 2019}

\begin{document}

\maketitle

\section{Introduction}

This document is licensed as GPL v2

As follows, Defining sub-routines

\section{Graphs}

Graphs are originally a mathematical structre, $G(V,E)$ where $V$ is a set of vertices and $E$ is a set of edges. where each edge $e$ connects two vertices $u$ and $v$.

%% Some formulas definitions

\paragraph{Edges set}

\paragraph{Vertices set}

Holds all the nodes that belong to the graph, but some might be isolated. a node $u$ is isolated if $!\exists e such that$


\subsection{Graph Properties}


\paragraph{Directed graphs}

a graph $G(V,E)$ is directed if $\exists e \in E where $

\paragraph{Undirected graphs}
\paragraph{Weighted}
\paragraph{Unweighted}

\section{Graph as a Data Structure}

%% a pseudo code data strcture of graph, edge and vertex

From a computer scinece prespective, graphs can be generalize to solve so many problems.

%% some well known problems

\subsection{Graph Traversal}

Given a graph $G(V,E)$, two fundamental traversing algorithms are BFS and DFS. Which both seem too obvious. BFS

\paragraph{Queues and priority}
Queue are \textit{FIFO}, good to use when having to track piriority.
%% TODO: add queue used functions

\paragraph{Breadth First Search and shortest path}
getting the shortest path
%% add example of BFS as algorithm

\paragraph{Depth First Search}
%% add exmaple DFS as algorithm

\section{Preparing the network}

\begin{quote}
  Given a graph $G(V,E)$, and a network flow $N(V, F, s, t, \upsilon)$ such that $F$ is an \textit{ordered sets} of flows, where the flow is an ordered set of edges $f_i$, representing flow \textit{i}, such that $\{i > 0, \forall f_i \in F, f_i \subseteq E$ and $f_i \leq |f_{i+1}|\}$. $s, t \in V$. \textit{The goal it synchronize $\upsilon$ unites, the quickest possible} way.
\end{quote}

Given $\sigma$ as the set of used flow, where at the beginning $\sigma = F$.

\begin{quote}

  \begin{itemize}
  \item [\textit{latency, $|f_i|$ }] it is the number of the edges a flow $f_i$ has, indeed $|f_i|$.

  \item [\textit{max delay, $\tau$}] the maximum delay of a network $N(V,F,s,t\upsilon)$ is the maximum latency among all other flows in set $F$, $\{f_i \in F, |f_i| \leq \tau\}$.

  \item [\textit{frequency}] the frequency of a flow $f_i$ in a network $N(V,F,s,t,\upsilon)$, noted as $freq(f_i)$, is how many times it synced compared to the  $\frac{\tau}{|f_i|}$

  \item The total number of flows, $c$ is $|F$, The \textit{upper boundary} of the flow network with $c$ flows, or the blob size $\omega$, is, $\sum\limits^{c}_{i=0}{|f_i|} $

  \item network state $\Delta$ is the number of units over upper boundary $\frac{\upsilon_{i}}{\omega}$

  \item $\upsilon_{i+1}$ is a series of \textit{left overs}, remain after sending $\upsilon_{i}$ $\Delta$-times, where $\upsilon_{i+1}  \equiv \upsilon_{i}[\omega]$

  \item Given $\sigma$ as a set where, $\sigma \subseteq F$ and $i > 0, late(\sigma_i) < \upsilon$
  \end{itemize}

\end{quote}

\section{Flow synchronization}

\end{document}
